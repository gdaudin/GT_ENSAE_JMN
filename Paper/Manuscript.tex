\documentclass{article}
\usepackage{authblk}
\usepackage{natbib}
\usepackage[utf8]{inputenc}
\usepackage[T1]{fontenc}
\usepackage[pdftex]{graphicx}  
%Langue utilisée
\usepackage[english]{babel}
\usepackage[T1]{fontenc}
\emergencystretch=8em
\usepackage[margin=1in]{geometry}



\title{Comment: Inferring Trade Costs from Trade Booms and Trade Busts}

\author[1]{Guillaume Corlay}
\author[2]{Stéphane Dupraz}
\author[3]{Claire Labonne}
\author[1]{Anne Muller}
\author[4]{Céline Antonin}
\author[5]{Guillaume Daudin\footnote{Corresponding author: guillaume.daudin@dauphine.fr}}
\affil[1] {ENSAE, French National School of Statistics and Economic Administration, 3, avenue Pierre Larousse, 92245 Malakoff, FRANCE}
\affil[2] {Columbia University, 10025 New York NY, USA}
\affil[3] { Paris School of Economics / Université Paris 1 Panthéon Sorbonne -- ACPR - Banque de France, 75009, PARIS, FRANCE}
\affil[4] {Sciences Po, Observatoire Français des Conjonctures Économiques (OFCE), 75007, PARIS, FRANCE}
\affil[5] {Université Paris-Dauphine, PSL Research University, LEDa, 75016 PARIS, FRANCE
Université Paris-Dauphine, PSL Research University, LEDa, UMR [225], DIAL, 75016 PARIS, FRANCE
Sciences Po, Observatoire Français des Conjonctures Économiques (OFCE), 75007, PARIS, FRANCE}




%\date{March 2, 2012}

\begin{document}

\maketitle


\begin{abstract}
\cite{JMN2011} offer an alternative to price gaps to quantify
trade costs. Implementing a method which consists in deducing international
trade costs from trade flows, they argue that the reduction in
trade costs was the main driving force of trade growth
during the first globalization (1870-1913), whereas economic expansion was the main driving force during the second globalization (1950-2000). We argue
that this important result is  driven by the use of an \textit{ad hoc} aggregation
method. What \cite{JMN2011} capture is the 
difference in the relative starting trade of dyads experiencing faster trade growth in the first and second globalization.
More generally, we cast doubts on the possibility to reach
conclusions of such nature with a method that infers trade
costs from trade flows, and then uses these costs to explain
trade flows. We argue that it can only rephrase the information already
contained in openness ratios.
\end{abstract}


Keywords: Trade costs, globalization, gravity model, aggregation,
structure effect.

JEL Code: F14, N70


\section{Introduction}

Jacks and his coauthors offer in several papers  a method for
inferring international trade costs from trade flows.\footnote{The method is developed in \cite{JMN2008}, \cite{JMN2010}, \cite{NOVY} and \cite{JMN2011}. We will use this latter paper as a reference} Using the general equilibrium model of
\cite{AW2003}, they calculate trade costs (defined as all barriers to trade, notably transportation and transaction costs) and their
evolution during the first and second waves of globalization
(1870-1913 and 1950-2000) as well as the interwar period
(1921-1939) thanks to the impressive set of data they collected
on trade flows, GDP and exports between 27
countries\footnote{Argentina, Australia, Austria, Belgium,
Brazil, Canada, Denmark, France, Germany, Greece, India,
Indonesia, Italy, Japan, Mexico, the Netherlands, New Zealand,
Norway, the Philippines, Portugal, Spain, Sri Lanka, Sweden,
Switzerland, the United Kingdom, the United States, and
Uruguay. The data contain 130 country pairs.}. They provide a
decomposition of the growth of trade caused by the reduction in
trade costs and economic expansion. They use their computations to underline
a difference of nature between the two globalizations:

\begin{quote}
\emph{``Our results assign an overarching role for our trade
cost measure in the nineteenth century and the interwar trade
bust. In contrast, when explaining the post-World War II trade
boom, we identify a more muted role for the trade cost
measure.'' (p. 196)}.
\end{quote}

This is potentially an important result that sheds light on the globalization processes. However, that result is actually driven by an \textit{ad hoc}
method of aggregation that captures structure effects. The inferred difference between the two globalization comes from the fact that the dyads with the 
fastest growing trade in the first wave of globalization start with very small 
trade. This is not the case in the second wave of globalization.
Using a three-countries version of \cite{AW2003}'s model for
the aggregation of bilateral trade costs shows that there was no 
difference in nature between the two globalizations. More generally, we
cast doubt on the possibility to distinguish the effect of trade costs and economic expansion through an approach that relies only on the study of trade
flows. Deducing trade costs from trade flows, and then using
trade costs to explain trade flows is essentially a circular
reasoning. Therefore, Jacks \textit{et al.}'s approach cannot be an
alternative to traditional investigations of impediments to
trade, such as commodity price gaps.

This paper starts by presenting Jacks \textit{et al.}'s approach of the
measure of trade costs, and insists on its relevance at the
bilateral level to control for multilateral trade barriers in
gravity regressions (section \ref{1}). It then highlights that the
result on a difference of nature between the two globalizations
is paradoxical since it cannot be deduced from a comparison of the
evolution of openness ratios. Section \ref{2} shows that the
conclusion is only driven by the authors' \textit{ad hoc} aggregation
method. We propose a microfounded way to aggregate trade costs
and the puzzle fades away. Section \ref{3} explores the reasons
why Jacks \textit{et al.}'s aggregation technique ends up providing
different results for the two globalizations. We argue that
what Jacks \textit{et al.} misleadingly attribute to unequal trade costs
decreases between the two globalizations is instead a
difference in the distribution of trade growth over trading
partners.

\section{\label{1} Deducing trade costs from trade flows, explaining trade flows with trade costs}

Although it is consistent with many models of international
trade, Jacks, Meissner and Novy's work is primarily based on
the general equilibrium model framework of \cite{AW2003}. $n$
countries, each represented by a maximizing consumer, exchange
goods over one single period. In this Armington world,
production is not modelized\footnote{In the working paper
version of \cite{JMN2010}, the authors provide a version of the
model with production. The key equation is identical to the one
of the model without production.} and each country is initially
endowed with a differentiated representative good. Trade occurs
because of consumers' taste for diversity. The preferences of
all countries are assumed to be identical and modelized by a
Constant Elasticity of Substitution (CES) utility function.

\cite{AW2003} use this model to microfound gravity equations
and solve \cite{MAC}'s border puzzle by highlighting that
bilateral trade does not depend on bilateral trade barriers \textit{per
se}, but bilateral trade barriers relative to impediments to
trade with all other trading partners. \cite{AW2003} show that the
equilibrium imposes the following relation:

\begin{eqnarray}
x_{ij}=\frac{y_i y_j}{y^W}\left(\frac{t_{ij}}{P_i P_j}\right)^{1-\sigma}
\end{eqnarray}

where $x_{ij}$ are real exports from $i$ to $j$, $y_i$ is real
output of country $i$, $y^W$ is the world real output, $\sigma$
is the elasticity of substitution, $P_i$ is the price index in
country $i$ and can be interpreted as multilateral trade
barriers, and $t_{ij}$ is the trade costs factor between $i$
and $j$. Trade costs factors are assumed to be symmetric, i.e.
$t_{ij}$ = $t_{ji}$.

Jacks \textit{et al.} first depart from \cite{AW2003} by eliminating the
multilateral resistance variable ($P_i P_j$) from the gravity equation.
Following \cite{NOVY}, bilateral trade barriers are not
considered relative to multilateral trade barriers modelized by
the price index, but to domestic trade costs. In this case,
trade flows are no longer compared to outputs, but to internal
trade $x_{ii}$\footnote{Due to data limitations, the authors
use the relation $x_{ii}=GDP_i-EXPORTS_i$ to get internal
trade. We follow them. Concerns about the fact that GDP is
measured in added-value and exportations as gross value are
addressed in appendix B of \cite{JMN2011}.}. The equation above
becomes:

\begin{eqnarray}
\left( \frac{x_{ii} x_{jj}}{x_{ij} x_{ji}}\right)^{\frac{1}{2(\sigma-1)}} =\left(\frac{t_{ij} t_{ji}}{t_{ii} t_{jj}} \right)^{\frac{1}{2}} = 1+\tau_{ij} \label{keyeq}
\end{eqnarray}

The last equality defines $\tau_{ij}$, the geometric mean
of trade costs from country $i$ to $j$ and $j$ to $i$
\emph{relative to intra-national trade costs in countries $i$
and $j$} (no assumption of symmetry of bilateral trade costs is
imposed). It is the trade cost measure used by the authors.

This simple step offers a significant breakthrough in the
microfoundations of gravity equations. Comparing bilateral
trade flows to intranational trade allows to integrate
\cite{AW2003}'s caveat against omitted variable bias while
getting rid of multilateral trade barriers. This is important
when multilateral trade barriers cannot be estimated, e.g. when
we only have data on trade for one country. \cite{JMN2011} take
advantage of this feature in the fifth section of their article in
regressing their measure of trade costs between two countries
on a set of proxies for trade costs, such as the distance
between trade partners, tariffs, or the volatility of the
exchange rate. This is a very useful approach.\\

However, Jacks, Meissner and Novy also suggest in their article that
\cite{AW2003}'s model can be used to move away from the gravity
regression approach. Instead of explaining trade flows by
observable proxies for costs, they quantify the impact of all
impediments to trade: \emph{``We, therefore, infer trade costs
from trade flows. This approach allows us to capture the
combined magnitude of tariffs, transport costs, and all other
macroeconomic frictions that impede international market
integration but which are inherently difficult to observe. We
emphasize that this approach of inferring trade costs from
readily available trade data holds clear advantages for applied
research: the constraints on enumerating — let alone,
collecting data on — every individual trade cost element
even over short periods of time makes a direct accounting
approach impossible.''} (p.131).

Our point is that a method that uses trade flows
alone can teach us much about their determinants. When
departing from \cite{AW2003}'s multilateral resistances, the
arbitrage condition Jacks \textit{et al.} base their computation on is an
equality between a Marginal Rate of Substitution (MRS) and a
price ratio for a CES utility function, or more precisely the
product of two such equalities, one from the program of each
trade partner. Appendix \ref{A1} derives equation (\ref{keyeq})
this way without passing through multilateral resistance. The
inference of trade costs from trade flows is tantamount to
using the relation between quantities and prices given by a
demand curve derived from fully specified exogenous
preferences.

This is of course no reason in itself to disregard the method.
Yet it cannot be used to answer \emph{``the central
question of what drives trade booms and busts''}(p. 186) as the
authors do in section 6 of their article, because the reasoning is
circular: first calculate trade costs from trade flows, then
explain trade flows from the inferred trade costs.

Formally, the authors take the logarithm of the key equation of
their article (\ref{keyeq}) to decompose the product of
bilateral trade flows between $i$ and $j$ in four terms:

\begin{eqnarray}
ln (x_{ij}x_{ji})= 2 ln(y_i + y_j)+ ln (s_i s_j) + 2(1-\sigma)ln(1+\tau_{ij})+ \left(\frac{x_{ii} x_{jj}}{y_i y_j}\right)
\end{eqnarray}

Where $s_i=y_i/(y_i+y_j)$. By calculating the GDP-weighted
average of the evolution of these terms over the first and
second globalizations, the authors attribute the evolution of
trade flows to four components: output growth, increasing
income similarity, changes in trade costs and a fourth term
that can be interpreted as a trade diversion effect.

Since the trade costs measure is assimilated to trade flows
relative to domestic activity, they simply restate
 an information already contained in any measure of
trade flows relative to economic size, like an openness ratio
defined for instance as the ratio of exportations to GDP.
However the main conclusion in Jacks \textit{et al.}'s article, namely
that the first globalization was driven more by the decrease in
trade costs than the second one, requires the absence of circularity.
Indeed, when we decompose the level of
exportations of a country as the product of its GDP and its
openness ratio, we find results that are different from theirs (see table \ref{OR}).\footnote{We follow
the authors in interpreting log differences as percentages. However, one should keep in mind that given the
size of the changes, this is a very inexact approximation.} The increase in exportations is mainly explained by GDP growth for the large majority of the 27 countries in the
data. On average, this decomposition attributes 74\% (183/246) of
the growth in trade to the increase in GDP in the second
globalization and 60\% in the first globalization. This is very much in contrast to 
Jacks \textit{et al.}\footnote{Contributions of growth in income
similarity and of change in multilateral factors are negative.} :
 
\begin{quote}
\emph{``For the pre-World War I period, we find that declines
in the trade cost measure explain roughly 60\% of the growth in
global trade. [...] Conversely, we find that only 31\% of the
present-day global trade boom can be explained by the decline
in the trade cost measure. [...] The contribution of the two
trade booms suggests that major technological breakthrough in
the nineteenth century such as the steamship, the telegraph,
and refrigeration may have been relatively more important than
technological innovations in the second half of the twentieth
century such as containerization and enhanced handling
facilities.'' (p. 186)}.
\end{quote}


\begin{table}
\centering
\begin{tabular}{|r|r|r|r|r|r|r|}
\hline
 & \multicolumn{3}{|c|}{First Globalization 1870-1913} & \multicolumn{3}{|c|}{Second Globalization 1950-2000}\\
\hline
 & Exports & GDP & Openness Ratio & Exports & GDP & Openness Ratio\\
\hline
Argentina & 292 & 251 & 40 & 116 & 132 & -16\\
\hline
Austria & 89 & 102 & -13 & 379 & 185 & 194\\
\hline
Australia & 174 & 152 & 22 & 114 & 191 & -77\\
\hline
Belgium & 195 & 86 & 109 & 251 & 150 & 101\\
\hline
Brazil & 174 & 101 & 73 & 184 & 239 & -55\\
\hline
Canada & 233 & 170 & 64 & 258 & 192 & 66\\
\hline
Denmark & 200 & 113 & 87 & 230 & 142 & 88\\
\hline
France & 117 & 70 & 48 & 260 & 172 & 87\\
\hline
Germany & 168 & 119 & 48 & 391 & 175 & 216\\
\hline
Greece & 140 & 99 & 42 & 227 & 218 & 9\\
\hline
India & 147 & 41 & 105 & 161 & 216 & -54\\
\hline
Indonesia & 220 & 87 & 133 & 263 & 232 & 31\\
\hline
Italy & 151 & 83 & 69 & 335 & 188 & 147\\
\hline
Japan & 337 & 104 & 233 & 448 & 281 & 167\\
\hline
Mexico & 189 & 143 & 46 & 296 & 238 & 59\\
\hline
Netherlands & 230 & 92 & 138 & 298 & 173 & 125\\
\hline
New Zealand & 184 & 186 & -2 & 307 & 133 & 174\\
\hline
Norway & . & . & . & 70 & 181 & -111\\
\hline
Philippines & 218 & 92 & 126 & 237 & 213 & 24\\
\hline
Portugal & 74 & 57 & 17 & 319 & 208 & 111\\
\hline
Spain & 171 & 76 & 96 & 496 & 230 & 266\\
\hline
Sri Lanka & 172 & 92 & 81 & 39 & 201 & -161\\
\hline
Sweden & 150 & 92 & 58 & 240 & 134 & 106\\
\hline
Switzerland & 105 & 108 & -3 & 250 & 132 & 118\\
\hline
UK & 125 & 81 & 44 & 196 & 122 & 74\\
\hline
Uruguay & 261 & 165 & 96 & 48 & 94 & -46\\
\hline
USA & 208 & 166 & 42 & 241 & 170 & 72\\
\hline
 &  &  &  &  &  & \\
\hline
Average & \textbf{178} & \textbf{107} & \textbf{71} & \textbf{246} & \textbf{183} & \textbf{63}\\
\hline
\end{tabular}
\caption{Decomposition of the growth of exportations between
GDP growth and Openness ratio growth, log differences (interpreted as percentages).
Figures for Norway are not given because the dissolution of the
union between Norway and Sweden (1905) makes them
meaningless.}\label{OR}
\end{table}

This contrast between both decompositions of the growth of exports is at the center of the argument by Jack \textit{et al.} As the next section shows, it is actually driven by the \textit{ad hoc} way they aggregate trade costs.

\section{\label{2} \textit{Ad hoc} aggregation of trade costs}

Jacks \textit{et al.}'s conclusion on a difference of nature between the
two globalizations is based on an aggregate trade costs
measure that provides a summary statistic of the
evolution of trade costs across all dyads in the sample, as
well as for different regions of the world. To move from
bilateral costs $\tau_{ij}$ to an aggregate measure of trade
costs, the authors use an arithmetic mean over dyads, weighted
by the sum of the GDP of the two trade partners.

There is no justification for this aggregation method. We argue that
all the results they reach that are not a reformulation of the
evolution of the openness ratio during the two
globalizations come from this aggregation method.\\

\cite{AW2003}'s model can be used to calculate the aggregate
trade costs a single country faces in its trade with all
its trading partners. Indeed, using a two-countries model where
all trade partners of country $i$ are treated as one single
country provides an aggregate measure of trade costs that does
not rely on any \textit{ad hoc} aggregation method but instead on
microfoundations. Since all trade flows data are not available,
it is also possible to use a three-countries model, with
country $i$, its trading partners present in the data, and the
rest of the world.

Let us note $\upsilon_{i}$ the corresponding trade cost faced
by country $i$ with all its trading partners in the data. Its
expression is given by Jacks \textit{et al.}'s key equation
(\ref{keyeq}) applied to a three-countries model:

\begin{eqnarray}
1+\upsilon_{i}=\left(\frac{x_{ii} x_{pp}}{x_{ip}
x_{pi}}\right)^{\frac{1}{2(\sigma-1)}} \label{keyeq3model}
\end{eqnarray}

where $x_{ip}$ and $x_{pi}$ are respectively real exports and
imports of $i$ from and to its trade partners in the data, and
$x_{pp}$ is the volume of trade within and between the trading
partners present in the data. This `domestic trade' variable
now
includes cross-border trade.\\

Using this measure of trade costs, we provide the same
decomposition in four terms of the increase in trade flows
during the two globalizations as in the authors' article for
the countries with the most trade partners available in the
data (France (24), the UK (25), the USA (23)). We also provide
the unweighted and end-of-period-GDP-weighted averages for all
countries in the sample. Results are displayed in tables
\ref{glob1} and \ref{glob2} along with the results with the
Jacks \textit{et al.}'s method.\footnote{They include unweighted and
end-of-period-GPD-weighted averages over country dyads of
bilateral trade costs, such as provided in \cite{JMN2011}. We
also display for France, the UK and the USA the trade cost
measure averaged over trading partners faced by one country, as
the results are presented (with a smaller data set) in
\cite{JMN2008, JMN2010}. We also provide both averages of this
measure over all countries in the data set.}\\

\begin{table}
\centering
\begin{tabular}{|l|c|c|c|c|c|}
\hline
\vbox{\hbox{\strut 1870-1913}\hbox{\strut }\hbox{\strut}} & \vbox{\hbox{\strut Contribution}\hbox{\strut of growth in}\hbox{\strut output}} & \vbox{\hbox{\strut Contribution}\hbox{\strut of growth in}\hbox{\strut income similarity}} &\vbox{\hbox{\strut Contribution}\hbox{\strut of change in}\hbox{\strut trade cost measure}} & \vbox{\hbox{\strut Contribution}\hbox{\strut of change in}\hbox{\strut multilateral factors}} & \vbox{\hbox{\strut Average growth}\hbox{\strut of bilateral}\hbox{\strut trade flows}}\\
\hline
&\multicolumn{5}{|c|}{JMN 2011, unweighted}\\
\hline
& 195 & 3 & 251 & -14 & 435\\
\hline
    &\multicolumn{5}{|c|}{JMN 2011, GDP-weighted}\\
\hline
& \textbf{225} & \textbf{-11} & \textbf{290} & \textbf{-18} & \textbf{486}\\
\hline
 &  &  &  &  & \\
\hline
&\multicolumn{5}{|c|}{JMN by country, unweighted}\\
\hline
France & 154 & 24 & 168 & -17 & 329\\
\hline
UK & 168 & 18 & 122 & -18 & 290\\
\hline
USA & 312 & -51 & 322 & -15 & 568\\
\hline
Average & 222 & -7 & 263 & -17 & 461\\
\hline
 &   &  &  &  & \\
\hline
&\multicolumn{5}{|c|}{JMN by country, GDP-weighted} \\
\hline
France & 187 & -1 & 151 & -12 & 325\\
\hline
UK & 194 & 3 & 102 & -13 & 286\\
\hline
USA & 286 & -29 & 244 & -11 & 490\\
\hline
Average & 228 & -9 & 295 & -13 & 501\\
\hline
 &  &  &  &  & \\
\hline
&\multicolumn{5}{|c|}{3 countries model}\\
\hline
France & 205 & -30 & 84 & -8 & 251\\
\hline
UK & 204 & -18 & 92 & -9 & 269\\
\hline
USA & 201 & 48 & 123 & -8 & 364\\
\hline
\vbox{\hbox{\strut Unweighted}\hbox{\strut average}}  & 206 & 3 & 128 & -12 & 325\\
\hline
\vbox{\hbox{\strut GDP-weighted}\hbox{\strut average}}  & \textbf{203} & \textbf{8} & \textbf{163} & \textbf{-9} & \textbf{365}\\
\hline
\end{tabular}
\caption{Decomposition of the growth in international trade
(logarithms) with \textit{ad hoc} averages and a microfounded
aggregation method. First wave of globalization,
1870-1913.  \emph{JMN 2011} refers to the averaging over dyads,
\emph{JMN by country} by country refers to the averaging over
trading partners for one country, \emph{3 countries model}
refers to the aggregation method we offer.}\label{glob1}
\end{table}

\begin{table}
\centering
\begin{tabular}{|l|c|c|c|c|c|}
\hline
\vbox{\hbox{\strut 1950-2000}\hbox{\strut }\hbox{\strut}} & \vbox{\hbox{\strut Contribution}\hbox{\strut of growth in}\hbox{\strut output}} & \vbox{\hbox{\strut Contribution}\hbox{\strut of growth in}\hbox{\strut income similarity}} &\vbox{\hbox{\strut Contribution}\hbox{\strut of change in}\hbox{\strut trade cost measure}} & \vbox{\hbox{\strut Contribution}\hbox{\strut of change in}\hbox{\strut multilateral factors}} & \vbox{\hbox{\strut Average growth}\hbox{\strut of bilateral}\hbox{\strut trade flows}}\\
\hline
&\multicolumn{5}{|c|}{JMN 2011, unweighted}\\
\hline
 & 353 & 8 & 148 & -25 & 484\\
\hline
&\multicolumn{5}{|c|}{JMN 2011, GDP-weighted}\\
\hline
 & \textbf{350} & \textbf{3} & \textbf{137} & \textbf{-17} & \textbf{473}\\
\hline
 &  &  &  &  & \\
\hline
&\multicolumn{5}{|c|}{JMN by country, unweighted}\\
\hline
France & 355 & -1 & 201 & -29 & 526\\
\hline
UK & 280 & 29 & 54 & -25 & 338\\
\hline
USA & 343 & 11 & 112 & -21 & 445\\
\hline
Average & 349 & 9 & 124 & -23 & 459\\
\hline
 &  &  &  &  & \\
\hline
&\multicolumn{5}{|c|}{JMN by country, GDP-weighted}\\
\hline
France & 365 & 2 & 271 & -21 & 617\\
\hline
UK & 319 & 1 & 202 & -20 & 502\\
\hline
USA & 349 & 27 & 165 & -16 & 525\\
\hline
Average & 361 & 21 & 190 & -19 & 553\\
\hline
 &  &  &  &  & \\
\hline
&\multicolumn{5}{|c|}{3 countries model} \\
\hline
France & 371 & -12 & 259 & -16 & 602\\
\hline
UK & 372 & -59 & 122 & -15 & 420\\
\hline
USA & 372 & -7 & 195 & -12 & 548\\
\hline
\vbox{\hbox{\strut Unweighted}\hbox{\strut average}}  & 343 & 11 & 116 & -18 & 452\\
\hline
\vbox{\hbox{\strut GDP-weighted}\hbox{\strut average}}  & \textbf{361} & \textbf{15} & \textbf{183} & \textbf{-14} & \textbf{545}\\
\hline
\end{tabular}
\caption{Decomposition of the growth in international trade
(logarithms) with \textit{ad hoc} averages and a microfounded
aggregation method. Second wave of globalization, 1950-2000.}\label{glob2}
\end{table}

Tables \ref{glob1} and \ref{glob2} highlight how much the decomposition
between the decrease in trade costs and the income growth depends on the aggregation technique. When we use the microfounded aggregation method,\footnote{We selected the (GDP-weighted) average of our measure in order
to allow a clear comparison with Jacks \textit{et al.}'s results. There
is of course no rational for such a summary statistic, but the
results for France, the UK and the USA assure that the main
conclusion of this exercice does not depend on averaging over
countries.} 
growth in output is the main driving force behind both waves of growth in
international trade, contributing to about 56\% (203/365) in
the first wave and 66\% (361/545) in the second one. It is no coincidence
if these results are very similar to the ones in table \ref{OR}: 60\% (71/107) in the first globalization and 74\% (183/246) in the second one.
Insofar as a measure of
trade costs is defined residually as everything that explains
differences between domestic and international trade flows, it
is bound to reword the information contained in an openness
ratio.

\section{\label{3} Sensitivity of the trade cost measure to
structure effects}

Our explanation of why Jacks \textit{et al.}'s aggregate measure of
trade costs yields different conclusion 
from table \ref{OR} is because it is sensitive to structure
effects. To explain this idea, and for clarity purposes, let us consider
a word where partners are symmetric (domestic trade is equal in $i$ and $j$ and $x_ij$=$x_ji$). From equation \ref{keyeq}, we have:

\begin{eqnarray}
1+\tau_{ij}= \left(\frac{x_{ii}}{x_{ji}}\right)^{\frac{1}{\sigma-1}}\label{tausimplified}
\end{eqnarray}

This relation does not of course correspond to the arbitrage
condition of country $i$, which would involve supply prices,
but our aim here is to highlight the mechanism behind Jacks et
al.'s aggregation. Note $\overline{a_j}$ the arithmetic mean
of $a_j$ over $j$ ($\overline{a_j}=1/n \sum_{j=1}^{n}a_j$). One can
then compare the unweighted average of $\tau_{ij}$ according to equation 
(\ref{tausimplified}) and the measure derived from the
three-countries model in equation (\ref{keyeq3model}):

\begin{eqnarray}
1+\tau_i&=& x_{ii}^{\frac{1}{\sigma-1}} \left(\overline{{x_{ji}^{\frac{1}{1-\sigma}}}}\right) \label{tau}\\
1+\upsilon_i&=&x_{ii}^{\frac{1}{\sigma-1}} (n \times \overline{x_{ji}})^{\frac{1}{1-\sigma}}\label{upsilon}
\end{eqnarray}

Except for the factor $n$ (which is irrelevant since we are
concerned with the evolution of the trade costs index) the two
expressions (\ref{tau}) and (\ref{upsilon}) differ only by the
mean they use. On the one hand, $\upsilon_i$ uses the
arithmetic mean of imports $x_{ji}$. On the other hand,
$\tau_i$, because it uses an arithmetic mean over $\tau_{ij}$,
uses a mean power $1/(1-\sigma)$ of the imports,
$\left(\overline{x_{ji}^{\frac{1}{1-\sigma}}}\right)^{1-\sigma}$.
Appendix \ref{A2} establishes that the curvature
properties of the function $x \rightarrow
x^{\frac{1}{1-\sigma}}$, $\sigma>1$ tend to draw the growth of
$\tau_i$ towards the values of the growth of bilateral trade
costs incurred with small trading partners.\footnote{The `size' of a country is
used here to refer to the importance of the volume of its trade
with $i$ among trade partners of country $i$, regardless of its
population and GDP.} On the contrary,
$\upsilon_i$ puts more weight on big trade partners, consistent
with intuition and their higher share of imports.
%If the
%arithmetic mean is independent of the distribution of the
%importations over trade partners, the mean power $1/(1-\sigma)$
%is extremely sensitive to structure effects.

The authors partly correct this biais in the average of $\tau_{ij}$ by the use a end-of-period-GDP
weighted average. However, as tables \ref{glob1} and \ref{glob2} show,
weighting by the GDP of trade partners, besides not being
theoretically justified, does not provide an accurate
correction of the bias in the measure. One reason for this
failure is that the relationship between GDP and the importance
of trade flows is not systematic.\\

The average of the bilateral trade costs $\tau_{ij}$ captures both trade
costs and the distribution of trade costs relative to the size
of the trading partners. Through time, if dyads with small starting trade experience  faster growth of trade than others, the decline of trade costs measured by $\tau_i$ will be overestimated compared to the decline of $\upsilon_i$.
This fits the data. For both globalizations, figure
\ref{structure} plots the growth of trade flows (measured by
$\Delta ln (\sqrt{x_{ij} x_{ji}})$) as a function of the
initial value of trade (measured by the logarithm of the
geometric average of bilateral average $ln(\sqrt{x_{ij}
x_{ji}})$) for all dyads in the sample. This is also confirmed in the contrast between the rate of growth of exports
(table \ref{OR}) and the rate of growth of bilateral trade flows (table \ref{glob1} and \ref{glob2}), as export growth was relatively faster than bilateral trade growth in the first globalization compared to the second (486/178 > 473/246). It confirms that the pairs of countries that initially traded little together experienced faster growth of trade compared to other pairs in the first than in the second globalization.  This in an interesting result, but not the one presented by JMN.

\begin{figure}
\centering
\includegraphics{structure.eps}
\caption{Growth of trade flows (logarithms) depending on their initial value for the first (1870-1913) and second (1950-2000) globalizations (logarithms).}\label{structure}
\end{figure}

\section{\label{5} Conclusion}

Jacks, Meissner and Novy's method for inferring trade costs
from trade flows simply reformulates the evolution of the
openness ratio when it is used to calculate aggregate trade
costs. This is because it only relates the two through an
equality between MRS and price ratio. It appears more clearly
when comparing the \textit{ad hoc} aggregation of bilateral trade costs
to an aggregated measure of trade costs based on
\cite{AW2003}'s model. Still, if their method fails to offer a
full alternative to traditional investigations of trade costs,
such as commodity price gaps, it provides an improvement in the
microfoundation of gravity equations by substantially
simplifying the way of correcting the omitted variable bias.

This characteristic of the proposed measure of trade costs is
partly due to the all-inclusive definition of trade
costs used by the authors. In such a model, costs are anything
that causes consumptions from different countries' products not
to be equal. To be sure, it is essential to highlight that
trade costs cannot be reduced to tariffs or transportation
costs, and to insist on the need for a quantification of all
the impediments to international trade. But the concept of
trade costs loses part of its interest if there are no causal
alternatives to explain trade patterns. Such a definition is
therefore bound to reword the information given by trade
flows relative to output, such as contained in an openness
ratio.

\appendix

\section{\label{A1}Appendix: Deducing the measure of trade costs from an equality between a MRS and a price ratio}

We derive in this appendix the key equation in \cite{JMN2011}.
We do not start from \cite{AW2003}'s results as we want to
highlight it is an equality between MRS and price ratio, or
more precisely the product of two such equalities, one from the
program of each trade partner.

Let us note $C_{ki}$ the consumption by country $i$ of good from
region $k$, $\sigma$ the elasticity of substitution, and
$\beta_k$ a positive distribution parameter, preferences of
countries $i$ and $j$'s representative consumers are given by
their respective utility functions:

\begin{eqnarray*}
U_i=\left(\sum_k\beta_{k}^{\frac{1-\sigma}{\sigma}}C_{ki}^{\frac{\sigma-1}{\sigma}}\right)^{\frac{\sigma}{\sigma-1}}\\
U_j=\left(\sum_k\beta_{k}^{\frac{1-\sigma}{\sigma}}C_{kj}^{\frac{\sigma-1}{\sigma}}\right)^{\frac{\sigma}{\sigma-1}}
\end{eqnarray*}

Trade costs imply that prices are specific to the country of
consumption. We note $p_k$ the supply price of the producer in
country $k$ net of trade costs, and $p_{ki}$ the price of
region $k$ goods for region $i$ consumers. We define
$t_{ki}=p_{ki}/p_{k}$ the trade costs factor between $k$ and
$i$. Jacks \textit{et al.}'s variable of trade costs between $k$ and
$i$, $\tau_{ki}$ is then defined as the trade costs factor
between $k$ and $i$ relative to the domestic trade costs factor
$t_{ii}$: $\tau_{ki}=t_{ki}/t_{ii}-1$. In all that follows, we
use the same notations $\tau_{ki}$ for the geometric average of
$\tau_{ki}$ and $\tau_{ik}$. Symmetry of bilateral trade costs
does not need to be assumed.

Country $i$ seeks to maximize $U_i$ under the constraint
$\sum_k p_{ki} C_{ki}=y_i$, where $y_i$ is the output of
country $i$. The first order conditions equate the Marginal
Rates of Substitution (MRS) and the price ratio. In particular,
for the goods produced by $i$ and $j$:

\begin{eqnarray*}
MRS_{ji}=\frac{\frac{\partial U_i}{\partial C_{ji}}}{\frac{\partial
U_i}{\partial C_{ii}}}=
\left(\frac{\beta_j}{\beta_i}\right)^{\frac{1-\sigma}{\sigma}}\left(\frac{C_{ji}}{C_{ii}}\right)^{-\frac{1}{\sigma}}=\frac{p_{ji}}{p_{ii}}
\end{eqnarray*}

Or equivalently with nominal values of trade flows,
$x_{ji}=p_{ji}C_{ii}$ and $x_{ii}=p_{ii}C_{ii}$ :

\begin{eqnarray*}
\left(\frac{\beta_j}{\beta_i}\right)^{\frac{1-\sigma}{\sigma}}\left(\frac{x_{ji}}{x_{ii}}\right)^{-\frac{1}{\sigma}}=\left(\frac{p_{ji}}{p_{ii}}\right)^{\frac{\sigma-1}{\sigma}}
\end{eqnarray*}

We take the power $\sigma$ of this equation. Country $k$'s
first-order optimality condition is given by a permutation of
the indexes. We hence have the system of equations:

\begin{eqnarray*}
\left(\frac{\beta_j}{\beta_i}\right)^{1-\sigma}\left(\frac{x_{ii}}{x_{ji}}\right)=\left(\frac{p_{ji}}{p_{ii}}\right)^{\sigma-1}\\
\left(\frac{\beta_i}{\beta_j}\right)^{1-\sigma}\left(\frac{x_{jj}}{x_{ij}}\right)=\left(\frac{p_{ij}}{p_{jj}}\right)^{\sigma-1}
\end{eqnarray*}

We can get rid of the $\beta_j$ coefficients by taking the
side-by-side product of those two equations:

\begin{eqnarray*}
\left(\frac{x_{ii} x_{jj}}{x_{ij} x_{ji}}\right)=\left(\frac{p_{ij}
p_{ji}}{p_{ii} p_{jj}}\right)^{\sigma-1}
\end{eqnarray*}

The main insight from this derivation is that when introducing
the trade costs variable $\tau_{ij}$, the exporter's supply
prices disappear so that the trade cost factor can be expressed
as a function of trade flows only. We get the equation on which
Jacks, Meissner and Novy base their analysis:

\begin{eqnarray*}
x_{ij} x_{ji} =(x_{ii} x_{jj}) (1+\tau_{ij})^{2(1-\sigma)}
\end{eqnarray*}

\section{\label{A2}Appendix: Properties of means and structure effects}


Let $\phi$ be a continuous bijective function. We can define
$m_{\phi}$ the $\phi$-mean of a sample $(a_j)_{1\leq j\leq n}$
as the image by $\phi^{-1}$ of the arithmetic mean of the image
of the sample $(\phi(a_j))_{1\leq j\leq n}$. A common case is
when $\phi$ is a power function $x \rightarrow x^\alpha$. It
includes the arithmetic ($\alpha=1$), quadratic ($\alpha=2$)
and harmonic ($\alpha=-1$) means. Formally:

\begin{eqnarray*}
m_{\phi}=\phi^{-1}\left(\frac{1}{n}\sum_{j=1}^n \phi(a_j)\right)
\end{eqnarray*}

Properties of these means can be deduced from the monotonicity
and convexity properties of the function $\phi$. For instance,
if $\phi$ is convex, the convexity inequality gives:

\begin{eqnarray*}
\phi(m_{\phi})=\frac{1}{n}\sum_{j=1}^n\phi(a_j)\geq \phi\left(\frac{1}{n}\sum_{j=1}^n
a_j\right)
\end{eqnarray*}

If $\phi$, and hence $\phi^{-1}$, is also increasing, the
$\phi$-mean is superior to the arithmetic mean. Intuitively,
the convexity of $\phi$ gives, relative to the arithmetic mean,
more weight to the high values in the sample. Those results are
reversed if $\phi$ is concave and increasing or convex and
decreasing. Jacks, Meissner and Novy's aggregation method of
the $\tau_{kj}$ is tantamount to using a $\phi$-mean with
$\phi$ the power functions $\phi: x \rightarrow
x^{\frac{1}{1-\rho}}$, $\rho>1$ where the coherence with the
model imposes an arithmetic mean. Hence, since $\phi$ is
decreasing and convex, the substitution tends to overweight
small countries.\\

This is a static result. We are interested in the dynamic
behavior of trade flows. But Jacks \textit{et al.}'s measure also
overweights initially small trade partners in terms of the
increase in trade flows. This can be seen by calculating the
elasticities to the importations from country $k$ ($a_k$) of
the arithmetic mean ($m$) and of Jacks \textit{et al.}'s mean
($m_{\sigma}$) (keeping importations from other trade partners
$a_j$ constant). The elasticity of a generic mean $m_\sigma$ to
$a_k$ is given by
\mbox{$\varepsilon_{m_{\phi}}^k=\frac{1}{n}\frac{\phi'(a_k)}{\phi'(m_{\phi})}\frac{a_k}{m_{\phi}}$},
so that:

\begin{eqnarray*}
\varepsilon_m^k&=& \frac{1}{1+\sum\limits_{j \neq k} a_j/a_k}\\
\varepsilon_{m_{\sigma}}^k&=&\frac{1}{1+\sum\limits_{j\neq k} (a_j/a_k)^{\frac{1}{1-\sigma}}}
\end{eqnarray*}

The elasticity of the arithmetic mean $m$ to $a_k$ is
increasing in $a_k$ whereas the elasticity of $m_{\sigma}$ is
decreasing in $a_k$. A one-percent increase in the importations
from $k$ increases more the arithmetic mean of importations if
$k$ is initially an important importer, simply because $k$
represents a larger part of trade. But it is the opposite with
$m_{\sigma}$: the smaller the initial value of trade with $k$,
the bigger the impact of its growth on $m_{\sigma}$. Therefore,
Jacks \textit{et al.}'s measure is biased toward the growth rates of
costs incurred with small trade partners.

\bibliographystyle{aer}
\bibliography{bibliography}

\end{document}
