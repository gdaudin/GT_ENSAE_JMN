\documentclass{article}
\usepackage{authblk}
\usepackage{natbib}
\usepackage[utf8]{inputenc}
\usepackage[T1]{fontenc}
\usepackage[pdftex]{graphicx}  
%Langue utilisée
\usepackage[english]{babel}
\usepackage[T1]{fontenc}
\emergencystretch=8em
\usepackage[margin=1in]{geometry}
\usepackage{booktabs}
\usepackage{tabularx}
\usepackage{tablefootnote}
\usepackage{caption}
\newcommand{\Note}[1]{\caption*{Note: {#1}} }





\title{Comment: Inferring Trade Costs from Trade Booms and Trade Busts}

\author[1]{Guillaume Corlay}
\author[2]{Stéphane Dupraz}
\author[3]{Claire Labonne}
\author[1]{Anne Muller}
\author[4]{Céline Antonin}
\author[5]{Guillaume Daudin\footnote{Corresponding author: guillaume.daudin@dauphine.fr. The authors thank Marc Adam, Marcelo Olarreaga and anonymous referees for their comments and questions. Marc Adam pointed to us a slight data incoherence in Jacks \textit{et al.}'s .dta files that we have corrected in the following analysis (and helped us with the programming). The authors are the sole responsible of mistakes in the text. Stata and tex files are available at https://github.com/gdaudin/GT\_ENSAE\_JMN.}}
\affil[1] {ENSAE, French National School of Statistics and Economic Administration, 3, avenue Pierre Larousse, 92245 Malakoff, FRANCE}
\affil[2] {Columbia University, 10025 New York NY, USA}
\affil[3] { Paris School of Economics / Université Paris 1 Panthéon Sorbonne -- ACPR - Banque de France, 75009, PARIS, FRANCE}
\affil[4] {Sciences Po, Observatoire Français des Conjonctures Économiques (OFCE), 75007, PARIS, FRANCE}
\affil[5] {Université Paris-Dauphine, PSL Research University, LEDa, 75016 PARIS, FRANCE

Université Paris-Dauphine, PSL Research University, LEDa, UMR [225], DIAL, 75016 PARIS, FRANCE}




%\date{March 2, 2012}

\begin{document}

\maketitle


\begin{abstract}
\cite{JMN2011} offer a method to measure trade costs that relies exclusively on bilateral exports and GDP statistics.
They argue that the reduction in trade costs was the main driving force of trade growth during the first globalization (1870-1913), whereas economic expansion was the main driving force during the second globalization (1950-2000).
This potentially major result is driven by the use of an \textit{ad hoc} aggregation method of bilateral trade costs at the country and at the global levels.
What \cite{JMN2011} capture is that some pairs of countries experienced faster trade growth in the first globalization than in the second globalization.
More generally, we cast doubts on the possibility to reach conclusions on aggregate costs with a method that excludes \textit{a priori} changes in non-trade costs determinants of openness rates and hence can only rephrase the information contained in them.
\end{abstract}


Keywords: Trade costs, globalization, gravity model, aggregation,
structure effect.

JEL Code: F14, N70


\section{Introduction}

Jacks and his coauthors offer in several papers  an innovative method to measure trade costs.\footnote{The method is developed in \cite{JMN2008}, \cite{JMN2010}, \cite{NOVY} and \cite{JMN2011}. We will use this latter paper as a reference} Using the general equilibrium model of
\cite{AW2003}, they calculate trade costs (defined as all barriers to trade, notably transportation and transaction costs) and their
evolution during the first and second waves of globalization
(1870-1913 and 1950-2000) as well as the interwar period
(1921-1939) thanks to the impressive set of data they collected
on trade flows and GDP between 27
countries\footnote{Argentina, Australia, Austria, Belgium,
Brazil, Canada, Denmark, France, Germany, Greece, India,
Indonesia, Italy, Japan, Mexico, the Netherlands, New Zealand,
Norway, the Philippines, Portugal, Spain, Sri Lanka, Sweden,
Switzerland, the United Kingdom, the United States, and
Uruguay. The data contain 130 country pairs.}. They provide a
decomposition of the growth of trade caused by the reduction in
trade costs and economic expansion. They use their computations to underline
a difference of nature between the two globalizations:

\begin{quote}
\emph{``Our results assign an overarching role for our trade
cost measure in the nineteenth century and the interwar trade
bust. In contrast, when explaining the post-World War II trade
boom, we identify a more muted role for the trade cost
measure.'' (p. 196)}.
\end{quote}

This is potentially an important result that sheds light on the globalization processes.
However, this result is actually driven by an \textit{ad hoc} method of aggregation that captures structure effects.
The authors use a weighted arithmetic average of their measure of trade costs between country pairs (dyads) to compute country-specific trade costs. This is equivalent to computing a power mean with exponent $1/(1-\sigma)$ of the values of dyadic trade flows ($\sigma > 1$ is the elasticity of substitution).
In contrast, we show that an aggregation method theoretically rooted in \cite{AW2003}'s model would use a simple arithmetic mean of the value of the dyadic trade flows. 
Because $1/(1-\sigma)$ < 0, the importance of small dyadic trade flows in the computation of country-specific trade costs is too large in the authors' computations.
This is not compensated by the weight they use (end-of-period GDP). The authors' conclusion on the difference between the two globalization periods comes from the fact that the dyads with the fastest growing trade in the first wave of globalization start with very small trade; this is not the case in the second wave of globalization.
Indeed, we show that using our theory-based aggregation method, there is no difference in nature between the two globalizations.

More generally, we cast doubts on the possibility to distinguish between the impact of aggregate trade costs and the impact of aggregate economic expansion through an approach that relies solely on the study of trade flows and excludes \textit{a priori} other possible causes for the evolution of openness rates, like the evolution of vertical specialization and changes in the elasticity of substitution between domestic and foreign goods.
Therefore, Jacks \textit{et al.}'s approach cannot be an alternative to traditional investigations of impediments to trade at the global level, such as commodity price gaps.
It is much more useful to study bilateral trade costs, even though its usefulness is limited by the amount of structure that must be imposed on the data to use it. 
Still, without the added structure imposed on the data, deducing trade costs from trade flows, and then using trade costs to explain trade flows would essentially be a circular reasoning


We first present Jacks \textit{et al.}'s approach to the
measure of trade costs, and insist on its relevance at the
bilateral level to control for multilateral trade barriers in
gravity regressions. We then highlight that the
result on the difference of nature between the two globalizations
is paradoxical since it cannot be deduced from a comparison of the
evolution of openness ratios (section \ref{1}). Section \ref{2} shows that the
conclusion is only driven by the authors' \textit{ad hoc} aggregation
method. We propose a microfounded way to aggregate trade costs
and the puzzle fades away. Section \ref{3} explores the reasons
why Jacks \textit{et al.}'s aggregation technique ends up providing
different results for the two globalizations. We argue that
what Jacks \textit{et al.} misleadingly attribute to unequal trade costs
decreases between the two globalizations is instead a
difference in the distribution of trade growth over trading
dyads.

\section{\label{1} Deducing trade costs from trade flows}

Although it is consistent with many models of international
trade, Jacks, Meissner and Novy's work is primarily based on
the general equilibrium model framework of \cite{AW2003}. $n$
countries, each represented by a maximizing consumer, exchange
goods over one single period. In this Armington world,
production is not modelized and each country is initially
endowed with a differentiated representative good. Trade occurs
because of consumers' taste for diversity.\footnote{In the working paper
version of \cite{JMN2010}, the authors provide a version of the
model with production. The key equation is identical to the one
of the model without production.} The preferences of
all countries are assumed to be identical and modelized by a
Constant Elasticity of Substitution (CES) utility function.

\cite{AW2003} use this model to microfound gravity equations
and solve \cite{MAC}'s border puzzle by highlighting that
bilateral trade does not depend on bilateral trade barriers \textit{per
se}, but bilateral trade barriers relative to trade barriers with all other trading partners. \cite{AW2003} show that the
equilibrium imposes the following relation:

\begin{eqnarray}
x_{ij}=\frac{y_i y_j}{y^W}\left(\frac{t_{ij}}{P_i P_j}\right)^{1-\sigma}
\end{eqnarray}

where $x_{ij}$ are real exports from $i$ to $j$, $y_i$ is real
output of country $i$, $y^W$ is the world real output, $\sigma$
is the elasticity of substitution, $P_i$ is the price index in
country $i$ and can be interpreted as multilateral trade
barrier or resistance, and $t_{ij}$ is the trade costs factor between $i$
and $j$. Trade costs factors are assumed to be symmetric, i.e.
$t_{ij}$ = $t_{ji}$.

Jacks \textit{et al.} first depart from \cite{AW2003} by eliminating the
multilateral resistance variable ($P_i P_j$) from the gravity equation.
As in \cite{NOVY}, they use the Head-Ries index (\cite{HR2001}) to express bilateral trade barriers not relatively to multilateral trade barriers modelized by
the price index, but relatively to domestic trade costs. In this case,
trade flows are no longer compared to outputs, but to internal
trade $x_{ii}$.\footnote{Due to data limitations, the authors
use the relation $x_{ii}=GDP_i-EXPORTS_i$ to get internal
trade. We follow them. Concerns about the fact that GDP is
measured in value-added and exportations as gross value are
addressed in appendix B of \cite{JMN2011}.} The equation above
becomes:

\begin{eqnarray}
\left( \frac{x_{ii} x_{jj}}{x_{ij} x_{ji}}\right)^{\frac{1}{2(\sigma-1)}} =\left(\frac{t_{ij} t_{ji}}{t_{ii} t_{jj}} \right)^{\frac{1}{2}} = 1+\tau_{ij} \label{keyeq}
\end{eqnarray}

The last equality defines $\tau_{ij}$, trade costs from country $i$ to $j$ and $j$ to $i$ \emph{relative to intra-national trade costs in countries $i$ and $j$} (no assumption of symmetry of bilateral trade costs is imposed). It is the trade cost measure used by the authors.

When departing in this way from \cite{AW2003}'s multilateral resistances, the arbitrage condition Jacks \textit{et al.} base their computation on is an equality between a Marginal Rate of Substitution (MRS) and a price ratio for a CES utility function---or more precisely the product of two such equalities, one from the program of each trade partner. 
Appendix \ref{A1} derives equation (\ref{keyeq}) this way without passing through multilateral resistance.
The inference of trade costs from trade flows is tantamount to using the relation between quantities and prices given by a demand curve derived from fully specified exogenous preferences. This is of course no reason in itself to disregard the method.

Actually, this simple step offers a significant breakthrough in the microfoundations of gravity equations. 
Comparing bilateral trade flows to intranational trade allows to integrate \cite{AW2003}'s caveat against omitted variable bias while getting rid of multilateral trade barriers.
This is important when multilateral trade barriers cannot be estimated, e.g. when we only have data on trade for one country. \cite{JMN2011} take advantage of this feature in the fifth section of their article in regressing their measure of trade costs between two countries on a set of proxies for trade costs, such as the distance between trade partners, tariffs, or the volatility of the exchange rate. This is a very useful approach.\\

Jacks, Meissner and Novy also suggest in their article that \cite{AW2003}'s model can be used to move away from the gravity regression approach. Instead of explaining trade flows by observable proxies for costs, they quantify the impact of all impediments to trade from trade flows:
\emph{``We, therefore, infer trade costs from trade flows.
This approach allows us to capture the combined magnitude of tariffs, transport costs, and all other macroeconomic frictions that impede international market integration but which are inherently difficult to observe.
We emphasize that this approach of inferring trade costs from readily available trade data holds clear advantages for applied research: the constraints on enumerating — let alone, collecting data on — every individual trade cost element even over short periods of time makes a direct accounting approach impossible.''}(p.131).

Formally, the authors take the logarithm of the key equation of their article (\ref{keyeq}) to decompose the product of bilateral trade flows between $i$ and $j$ in four terms:

\begin{eqnarray}
ln (x_{ij}x_{ji})= 2 ln(y_i + y_j)+ ln (s_i s_j) + 2(1-\sigma)ln(1+\tau_{ij})+ \left(\frac{x_{ii} x_{jj}}{y_i y_j}\right)
\end{eqnarray}

where $s_i=y_i/(y_i+y_j)$.\\

The authors hence attribute the evolution of bilateral trade flows to four components: output growth, increasing total income similarity, changes in trade costs and a trade diversion effect.

The method consists in determining the evolution of one unobservable variable (trade costs) based on the evolution of two groups of observable variables (trade flows and output) and a number of hypotheses on preferences and parameters ($\sigma$ and the relationship between $x_{ii}$, output and trade). A method that uses data on trade flows and output alone can only teach us so much about their determinants since non-trade costs determinants of trade flows (e.g. the elasticity of substitution and the importance of vertical specialization) are treated as parameters. If one assumes that preferences and parameters do not change (though these points are partly addressed  in appendix B of  \cite{JMN2011}), all changes in trade costs must restate information contained in changes in the measure of trade flows relative to economic size. Despite this reliance on a priori hypotheses, or structure, it is a useful approach at the bilateral level because there no simple indicator of bilateral openness. At the agregate level, there are obvious simple indicators of openness (e.g. openness ratio defined as the ratio of exportations to GDP), and the result of \cite{JMN2011} should restate it.

Yet, Jacks \emph{et al.}'s results seem to show otherwise. When one decomposes the level of exportations of a country simply as the product of its GDP and its openness ratio (see table \ref{OR}) one finds that for both globalizations the increase in exportations is mainly explained by GDP growth for the large majority of the 27 countries in the data.\footnote{We follow the authors in interpreting log differences as percentages. However, one should keep in mind that given the size of the changes, this is a very inexact approximation.}
On average, this decomposition attributes 74\% (183/246) of the growth in trade to the increase in GDP in the second globalization and 62\% in the first globalization.
This is very much in contrast to  Jacks \textit{et al.}\footnote{Contributions of growth in income similarity and of change in multilateral factors are negative.} :
 
\begin{quote}
\emph{``For the pre-World War I period, we find that declines in the trade cost measure explain roughly 60\% of the growth in global trade. [...] Conversely, we find that only 31\% of the present-day global trade boom can be explained by the decline in the trade cost measure. [...] The contribution of the two trade booms suggests that major technological breakthrough in the nineteenth century such as the steamship, the telegraph, and refrigeration may have been relatively more important than technological innovations in the second half of the twentieth century such as containerization and enhanced handling facilities.'' (p. 186)}.
\end{quote}

This contrast between both decompositions of the growth of exports is at the center of the argument by Jack \textit{et al.}.
It is surprising because it is not clear what is added by their method to the examination of openness rates.
As the next section shows, it is actually driven by the \textit{ad hoc} way they aggregate their measure of bilateral trade costs.
\begin{table}[tbp] \centering
\newcolumntype{C}{>{\centering\arraybackslash}X}

\begin{tabularx}{\textwidth}{lCCCCCC}

\toprule
& \multicolumn{3}{c}{First Globalization 1870-1913} & \multicolumn{3}{c}{Second Globalization 1950-2000}\\
{}&{Exports}&{GDP}&{Openness Ratio}&{Exports}&{GDP}&{Openness Ratio} \tabularnewline
\midrule\addlinespace[1.5ex]
ARG&292&251&40&116&132&-16 \tabularnewline
AUH&89&102&-13&379&185&194 \tabularnewline
AUS&174&152&22&114&191&-77 \tabularnewline
BEL&195&86&109&251&150&101 \tabularnewline
BRA&174&101&73&184&239&-55 \tabularnewline
CAN&233&170&64&258&192&66 \tabularnewline
DEN&200&113&87&230&142&88 \tabularnewline
FRA&117&70&48&260&172&87 \tabularnewline
GER&168&119&48&391&175&216 \tabularnewline
GRE&140&99&42&227&218&9 \tabularnewline
IND&147&41&105&161&216&-54 \tabularnewline
INN&220&87&133&263&232&31 \tabularnewline
ITA&151&83&69&335&188&147 \tabularnewline
JAP&337&104&233&448&281&167 \tabularnewline
MEX&189&143&46&296&238&59 \tabularnewline
NET&230&92&138&298&173&125 \tabularnewline
NEW&184&186&-2&307&133&174 \tabularnewline
NOR&&&&70&181&-111 \tabularnewline
PHI&218&92&126&237&213&24 \tabularnewline
POR&74&57&17&319&208&111 \tabularnewline
SPA&171&76&96&496&230&266 \tabularnewline
SRI&172&92&81&39&201&-161 \tabularnewline
SWE&150&92&58&240&134&106 \tabularnewline
SWI&105&108&-3&250&132&118 \tabularnewline
UK&125&81&44&196&122&74 \tabularnewline
URU&261&165&96&48&94&-46 \tabularnewline
USA&208&166&42&241&170&72 \tabularnewline
\textbf{Average}&\textbf{182}&\textbf{113}&\textbf{69}&\textbf{246}&\textbf{183}&\textbf{63} \tabularnewline
\bottomrule \addlinespace[1.5ex]

\end{tabularx}
\caption{Decomposition of the growth of exportations between
GDP growth and Openness ratio growth, log differences (interpreted as percentages).}
\Note{Figures for Norway are not given because the dissolution of the
union between Norway and Sweden (1905) makes them
meaningless.}\label{OR}
\end{table}



\section{\label{2} \textit{Ad hoc} aggregation of trade costs}

Jacks \textit{et al.}'s conclusion on a difference of nature between the
two globalizations is based on an aggregate trade costs
measure that provides a summary statistic of the
evolution of trade costs across all dyads in the sample. To move from
bilateral costs $\tau_{ij}$ to an aggregate measure of trade
costs, the authors use an arithmetic mean over dyads, weighted
by the sum of the GDP of the two trade partners.

There is no justification for this aggregation method. We argue that
all the agregate results they reach that are not a reformulation of the
evolution of the openness ratio during the two
globalizations come from this aggregation method.\\

To show this point, we reproduce Jacks \emph{et al.}'s decomposition using instead an aggregation method theoretically derived from \cite{AW2003}'s model. We calculate the aggregate trade costs a single country faces with all its trading partners by considering the two-country version of the model where all trade partners of country $i$ are treated as a unique one.

%The drawback of this method is that it is very data-intensive as it requires the knowledge of all bilateral trade flows between the countries in the data, whereas Jacks \emph{et al.}'s method  

%The last sentence ""The drawback of..." seemed peripheral and a distraction from the main line of thought? Besides, it is not seem more data-intensive than Jacks et al's measure SD
%Well, theirs is less data intensive as it does not requires the full population of bilateral trade flows. But I agree it can be treated as a distraction GD


Let us note $\upsilon_{i}$ the corresponding trade cost faced by country $i$ with all its trading partners in the data. Its expression is given by Jacks \textit{et al.}'s key equation (\ref{keyeq}) applied to a two-country model:

\begin{eqnarray}
1+\upsilon_{i}= \left(\frac{x_{ii} x_{pp}}{x_{ip}
x_{pi}}\right)^{\frac{1}{2(\sigma-1)}}
=\left(\frac{x_{ii} \sum_{j \neq i, k \neq i } x_{jk}}{\sum_{j \neq i } x_{ij}\sum_{j \neq i} x_{ji}}\right)^{\frac{1}{2(\sigma-1)}}   \label{keyeq2model}
\end{eqnarray}

where $x_{ip}$ and $x_{pi}$ are respectively real exports and
imports of $i$ from and to its trade partners in the data, and
$x_{pp}$ is the volume of trade within and between the trading
partners present in the data. This `domestic trade' variable
now
includes cross-border trade. Compare this with Jacks \textit{et al.}'s  method to measure the mean trade cost faced by a country : 
\begin{eqnarray}
1+\tau_i&=& \sum_{j \neq i}  {\frac{y_i}{\sum_{j \neq i} y_i} \left(1+\tau_{ij}\right)}
= \sum_{j \neq i}  {\frac{y_i}{\sum_{j \neq i} y_i} \left( \frac{x_{ii} x_{jj}}{x_{ij} x_{ji}}\right)^{\frac{1}{2(\sigma-1)}} }
 \label{jackssection3}
\end{eqnarray}

Tables \ref{glob1} and \ref{glob2} provide the decomposition in four terms of the increase in trade flows during the two globalizations for our and Jacks \textit{et al.}'s aggregation methods, for the countries with the most trade partners available in the data (France (24), the UK (25), the USA (23)).
This requires studying only trade between Belgium, Canada, France, Germany, India, Italy, Spain, United Kingdom and United States of America, as this is the largest set in the data where all bilateral trade flows are available. 
The tables also provide the unweighted and end-of-period-GDP-weighted averages for all countries in the sample.\footnote{The tables include unweighted and
	end-of-period-GPD-weighted averages over country dyads of
	bilateral trade costs, such as provided in \cite{JMN2011}.
	We also display for France, the UK and the USA the trade cost measure averaged over trading partners faced by one country, as the results are presented (with a smaller data set) in
	\cite{JMN2008, JMN2010}.
	We also provide both averages of this measure over all countries in the data set.}\\




\begin{table}[tbp] \centering
\newcolumntype{C}{>{\centering\arraybackslash}X}

\begin{tabularx}{\textwidth}{lCCCCC}

\toprule
{1870-1913}&{Contribution of growth in output}&{Contribution of growth in income similarity}&{Contribution of change in trade cost measure}&{Contribution of change in multilateral factors}&{Average growth of bilateral trade flows} \tabularnewline
\midrule\addlinespace[1.5ex]
&\multicolumn{5}{c}{JMN 2011} \tabularnewline
Unweighted&193&5&283&-15&466 \tabularnewline
\textbf{GDP-weighted}&\textbf{215}&\textbf{-6}&\textbf{275}&\textbf{-13}&\textbf{470} \tabularnewline

\midrule &\multicolumn{5}{c}{JMN by country, unweighted} \tabularnewline
FRA&165&7&122&-13&281 \tabularnewline
UK&175&7&96&-14&264 \tabularnewline
USA&289&-33&195&-11&441 \tabularnewline
Average&216&-7&272&-12&469 \tabularnewline
\midrule &\multicolumn{5}{c}{JMN by country, GDP-weighted} \tabularnewline
FRA&196&-11&127&-10&302 \tabularnewline
UK&201&-4&69&-11&256 \tabularnewline
USA&274&-25&192&-8&433 \tabularnewline
Average&227&-12&254&-10&460 \tabularnewline
\midrule &\multicolumn{5}{c}{Our method} \tabularnewline
FRA&202&-27&77&-6&245 \tabularnewline
UK&202&-16&67&-8&244 \tabularnewline
USA&202&43&89&-7&327 \tabularnewline
Unweighted average&202&-3&137&-7&329 \tabularnewline
\textbf{GDP-weighted average}&\textbf{202}&\textbf{13}&\textbf{126}&\textbf{-7}&\textbf{333} \tabularnewline
\bottomrule \addlinespace[1.5ex]

\end{tabularx}

\caption{Decomposition of the growth in international trade
(logarithms) with \textit{ad hoc} averages and a microfounded
aggregation method. First wave of globalization,
1870-1913. }
\Note {\emph{JMN 2011} refers to the averaging over dyads,
\emph{JMN by country} by country refers to the averaging over
trading partners for one country, \emph{our method}
refers to the aggregation method we offer.

The contribution of output growth is constant for all countries in our method because it is measured as the growth of total world output}
\label{glob1}
\end{table}






\begin{table}[tbp] \centering
\newcolumntype{C}{>{\centering\arraybackslash}X}

\begin{tabularx}{\textwidth}{lCCCCC}

\toprule
{1950-2000}&{Contribution of growth in output}&{Contribution of growth in income similarity}&{Contribution of change in trade cost measure}&{Contribution of change in multilateral factors}&{Average growth of bilateral trade flows} \tabularnewline
\midrule\addlinespace[1.5ex]
&\multicolumn{5}{c}{JMN 2011} \tabularnewline
Unweighted&355&4&284&-35&608 \tabularnewline
\textbf{GDP-weighted}&\textbf{350}&\textbf{6}&\textbf{230}&\textbf{-28}&\textbf{558} \tabularnewline
\midrule &\multicolumn{5}{c}{JMN by country, unweighted} \tabularnewline
FRA&350&3&318&-30&641 \tabularnewline
UK&297&11&227&-29&507 \tabularnewline
USA&341&9&180&-23&507 \tabularnewline
Average&350&6&224&-27&554 \tabularnewline
\midrule &\multicolumn{5}{c}{JMN by country, GDP-weighted} \tabularnewline
FRA&347&2&259&-21&587 \tabularnewline
UK&315&-12&219&-20&502 \tabularnewline
USA&342&10&161&-18&495 \tabularnewline
Average&347&9&184&-21&519 \tabularnewline
\midrule &\multicolumn{5}{c}{Our method} \tabularnewline
FRA&347&-1&318&-16&648 \tabularnewline
UK&347&-46&237&-15&522 \tabularnewline
USA&347&0&190&-11&526 \tabularnewline
Unweighted average&347&7&254&-15&592 \tabularnewline
\textbf{GDP-weighted average}&\textbf{347}&\textbf{2}&\textbf{202}&\textbf{-13}&\textbf{538} \tabularnewline
\bottomrule \addlinespace[1.5ex]

\end{tabularx}
\caption{Decomposition of the growth in international trade
(logarithms) with \textit{ad hoc} averages and a microfounded
aggregation method. Second wave of globalization, 1950-2000.}
\Note {\emph{JMN 2011} refers to the averaging over dyads,
\emph{JMN by country} by country refers to the averaging over
trading partners for one country, \emph{our method}
refers to the aggregation method we offer.

The contribution of output growth is constant for all countries in our method because it is measured as the growth of total world output}
\label{glob2}
\end{table}

Comparing the results for the two methods highlights how much the decomposition
between the decrease in trade costs and income growth depends on the aggregation method. When we use the microfounded aggregation method, growth in output is the main driving force behind both waves of growth in international trade, contributing to about 60\% (202/333) in the first wave and 66\% (361/545) in the second one.\footnote{We selected the (GDP-weighted) average of our measure in order
	to allow a clear comparison with Jacks \textit{et al.}'s results. There
	is of course no rational for such a summary statistic, but the
	results for France, the UK and the USA assure that the main
	conclusion of this exercice does not depend on averaging over
	countries.} These results are similar to the decomposition of trade flows between output and openness ratio in table \ref{OR}---62\% and 74\%---despite the sample change. Contrast this with Jack  \textit{et al.}'s results, respectively 46\% and 63\%.


\section{\label{3} Sensitivity of the trade cost measure to
structure effects}

The reason why Jacks \textit{et al.}'s aggregate measure of trade costs yields different conclusions compared to the microfounded method (and openness rates) is its sensitivity to structure
effects. To explain this idea, and for clarity purposes, let us move to a world of symmetric partners: domestic trade is equal in $i$ and $j$, $x_{ii}=x_{jj}$, and imports are equal to exports, $x_{ij}$=$x_{ji}$. (Obviously, trade flows $x_{ij}$ can still differ across trading partners---otherwise there would be no aggregation problem). Equation (\ref{keyeq}) gives: 
% The last sentence "Dyadic bilateral trade flows are not equal because some dyads are further apart than others." seemed out of context?

\begin{eqnarray}
1+\tau_{ij}= \left(\frac{x_{ii}}{x_{ji}}\right)^{\frac{1}{\sigma-1}}\label{tausimplified}
\end{eqnarray}

Note $\overline{a_j}$ the arithmetic mean
of $a_j$ over $j$ ($\overline{a_j}=1/n \sum_{j=1}^{n}a_j$). One can
write the unweighted average of $\tau_{ij}$ according to equation 
(\ref{tausimplified}) and the measure derived from the
two-country model in equation (\ref{keyeq2model}) as:

\begin{eqnarray}
1+\tau_i&=& x_{ii}^{\frac{1}{\sigma-1}} \left(\overline{{x_{ji}^{\frac{1}{1-\sigma}}}}\right) \label{tau}\\
1+\upsilon_i&=&x_{ii}^{\frac{1}{\sigma-1}} (n \times \overline{x_{ji}})^{\frac{1}{1-\sigma}}\label{upsilon}
\end{eqnarray}

Except for the factor $n$ (which is irrelevant since we are concerned with the evolution of the trade costs index) the two equations (\ref{tau}) and (\ref{upsilon}) differ only by the mean they use.
Equation (\ref{upsilon}) uses the arithmetic mean of trade flows $x_{ji}$. Equation (\ref{tau}), because it uses an arithmetic mean over $\tau_{ij}$, uses a power mean of exponent  $1/(1-\sigma)$ of the value of trade flows, $\left(\overline{x_{ji}^{\frac{1}{1-\sigma}}}\right)^{1-\sigma}$.
Appendix \ref{A2} establishes that the curvature properties of the function $x \rightarrow x^{\frac{1}{1-\sigma}}$, $\sigma>1$ tend to draw the growth of $\tau_i$ towards the values incurred with the trading partners with which $i$ does not trade much (small $x_{ij})$.
On the contrary, $\upsilon_i$ puts more weight on the trading partners with big trade flows, simply because they account for a larger part of trade.
%If the
%arithmetic mean is independent of the distribution of the
%importations over trade partners, the mean power $1/(1-\sigma)$
%is extremely sensitive to structure effects.

Let us now move away from our symmetric world. Jacks  \textit{et al.} partially correct that bias by using a end-of-period-GDP weighted average to compute the average of  $\tau_{ij}$.
However, as tables \ref{glob1} and \ref{glob2} show, weighting by the GDP of trade partners, besides not being theoretically justified, does not provide an accurate correction of the bias in the measure.
One reason for this failure is that the relationship between GDP and the importance of trade flows is not systematic.

For these reasons, Jacks  \textit{et al.}'s average of the bilateral trade costs $\tau_{ij}$ captures both the evolution of trade costs and the distribution of the evolution of trade costs over dyad sizes.
As a result, if dyads with small starting trade experience faster growth of trade than others, the decline of trade costs measured by $\tau_i$ will be overestimated compared to the decline of $\upsilon_i$.
This fits the data and can explain the difference they find between the first and the second globalization. For both globalizations, figure \ref{structure} plots the growth of trade flows (measured by $\Delta ln (\sqrt{x_{ij} x_{ji}})$) as a function of the initial value of trade (measured by the logarithm of the geometric average of bilateral average $ln(\sqrt{x_{ij} x_{ji}})$) for all dyads in the sample.\\

To sum up : in the first globalization, pairs of countries that initially traded little together experienced relative faster trade growth. This leads Jacks  \textit{et al.} to overestimate the role of trade costs in the first globalization.

\begin{figure}
\centering
\includegraphics{structure.eps}
\caption{Growth of trade flows (logarithms) depending on their initial value for the first (1870-1913) and second (1950-2000) globalizations (logarithms).}
\Note {The correlation for 1870-1913 is -0.635, with 95\% confidence interval: [-0.728;-0.520]

The correlation for 1950-2000 is -0.312, with 95\% confidence interval: [-0.459;-0.147]

These correlation coefficients are statistically significantly different.}
\label{structure}
\end{figure}

\section{\label{5} Conclusion}

Jacks, Meissner and Novy's method for inferring trade costs
from trade flows simply reformulates the evolution of the
openness ratio when it is used to calculate aggregate trade
costs. Namely it only relates the two through an
equality between the MRS and the price ratio. It appears more clearly
when replacing the \textit{ad hoc} aggregation of bilateral trade costs
with an aggregation method directly rooted in \cite{AW2003}'s model. Still, if their method fails to offer a
full alternative to traditional investigations of trade costs,
such as commodity price gaps, it provides an improvement in the
microfoundation of gravity equations by substantially
simplifying the way of correcting the omitted variable bias.

This characteristic of the proposed measure of trade costs is
partly due to the all-inclusive definition of trade
costs used by the authors. In such a model, costs are anything
that causes consumption flows from different countries' products not
to be equal. We agree that it is essential to highlight that
trade costs cannot be reduced to tariffs or transportation
costs, and to insist on the need for a quantification of all
the impediments to international trade. But the concept of
trade costs loses part of its interest if there are no causal
alternatives to explain changing trade patterns. Such a definition is
therefore bound to reword the information given by trade
flows relative to output, such as contained in an openness
ratio.

\appendix

\section{\label{A1}Appendix: Deducing the measure of trade costs from an equality between a MRS and a price ratio}

We derive in this appendix the key equation in \cite{JMN2011}.
We do not start from \cite{AW2003}'s results as we want to
highlight it is an equality between MRS and price ratio, or
more precisely the product of two such equalities, one from the
program of each trade partner.

Let us note $C_{ki}$ the consumption by country $i$ of good from
region $k$, $\sigma$ the elasticity of substitution, and
$\beta_k$ a positive distribution parameter, preferences of
countries $i$ and $j$'s representative consumers are given by
their respective utility functions:

\begin{eqnarray*}
U_i=\left(\sum_k\beta_{k}^{\frac{1-\sigma}{\sigma}}C_{ki}^{\frac{\sigma-1}{\sigma}}\right)^{\frac{\sigma}{\sigma-1}}\\
U_j=\left(\sum_k\beta_{k}^{\frac{1-\sigma}{\sigma}}C_{kj}^{\frac{\sigma-1}{\sigma}}\right)^{\frac{\sigma}{\sigma-1}}
\end{eqnarray*}

Trade costs imply that prices are specific to the country of
consumption. We note $p_k$ the supply price of the producer in
country $k$ net of trade costs, and $p_{ki}$ the price of
region $k$ goods for region $i$ consumers. We define
$t_{ki}=p_{ki}/p_{k}$ the trade costs factor between $k$ and
$i$. Jacks \textit{et al.}'s variable of trade costs between $k$ and
$i$, $\tau_{ki}$ is then defined as the trade costs factor
between $k$ and $i$ relative to the domestic trade costs factor
$t_{ii}$: $\tau_{ki}=t_{ki}/t_{ii}-1$. In all that follows, we
use the same notations $\tau_{ki}$ for the geometric average of
$\tau_{ki}$ and $\tau_{ik}$. Symmetry of bilateral trade costs
does not need to be assumed.

Country $i$ seeks to maximize $U_i$ under the constraint
$\sum_k p_{ki} C_{ki}=y_i$, where $y_i$ is the output of
country $i$. The first order conditions equate the Marginal
Rates of Substitution (MRS) and the price ratio. In particular,
for the goods produced by $i$ and $j$:

\begin{eqnarray*}
MRS_{ji}=\frac{\frac{\partial U_i}{\partial C_{ji}}}{\frac{\partial
U_i}{\partial C_{ii}}}=
\left(\frac{\beta_j}{\beta_i}\right)^{\frac{1-\sigma}{\sigma}}\left(\frac{C_{ji}}{C_{ii}}\right)^{-\frac{1}{\sigma}}=\frac{p_{ji}}{p_{ii}}
\end{eqnarray*}

Or equivalently with nominal values of trade flows,
$x_{ji}=p_{ji}C_{ii}$ and $x_{ii}=p_{ii}C_{ii}$ :

\begin{eqnarray*}
\left(\frac{\beta_j}{\beta_i}\right)^{\frac{1-\sigma}{\sigma}}\left(\frac{x_{ji}}{x_{ii}}\right)^{-\frac{1}{\sigma}}=\left(\frac{p_{ji}}{p_{ii}}\right)^{\frac{\sigma-1}{\sigma}}
\end{eqnarray*}

We take the power $\sigma$ of this equation. Country $k$'s
first-order optimality condition is given by a permutation of
the indexes. We hence have the system of equations:

\begin{eqnarray*}
\left(\frac{\beta_j}{\beta_i}\right)^{1-\sigma}\left(\frac{x_{ii}}{x_{ji}}\right)=\left(\frac{p_{ji}}{p_{ii}}\right)^{\sigma-1}\\
\left(\frac{\beta_i}{\beta_j}\right)^{1-\sigma}\left(\frac{x_{jj}}{x_{ij}}\right)=\left(\frac{p_{ij}}{p_{jj}}\right)^{\sigma-1}
\end{eqnarray*}

We can get rid of the $\beta_j$ coefficients by taking the
side-by-side product of those two equations:

\begin{eqnarray*}
\left(\frac{x_{ii} x_{jj}}{x_{ij} x_{ji}}\right)=\left(\frac{p_{ij}
p_{ji}}{p_{ii} p_{jj}}\right)^{\sigma-1}
\end{eqnarray*}

The main insight from this derivation is that when introducing
the trade costs variable $\tau_{ij}$, the exporter's supply
prices disappear so that the trade cost factor can be expressed
as a function of trade flows only. We get the equation on which
Jacks, Meissner and Novy base their analysis:

\begin{eqnarray*}
x_{ij} x_{ji} =(x_{ii} x_{jj}) (1+\tau_{ij})^{2(1-\sigma)}
\end{eqnarray*}

\section{\label{A2}Appendix: Properties of means and structure effects}


Let $\phi$ be a continuous bijective function. We can define
$m_{\phi}$ the $\phi$-mean of a sample $(a_j)_{1\leq j\leq n}$
as the image by $\phi^{-1}$ of the arithmetic mean of the image
of the sample $(\phi(a_j))_{1\leq j\leq n}$. A common case is
when $\phi$ is a power function $x \rightarrow x^\alpha$. It
includes the arithmetic ($\alpha=1$), quadratic ($\alpha=2$)
and harmonic ($\alpha=-1$) means. Formally:

\begin{eqnarray*}
m_{\phi}=\phi^{-1}\left(\frac{1}{n}\sum_{j=1}^n \phi(a_j)\right)
\end{eqnarray*}

Properties of these means can be deduced from the monotonicity
and convexity properties of the function $\phi$. For instance,
if $\phi$ is convex, Jensen inequality gives:

\begin{eqnarray*}
\phi(m_{\phi})=\frac{1}{n}\sum_{j=1}^n\phi(a_j)\geq \phi\left(\frac{1}{n}\sum_{j=1}^n
a_j\right)
\end{eqnarray*}

If $\phi$, and hence $\phi^{-1}$, is also increasing, the
$\phi$-mean is greater than the arithmetic mean. Intuitively,
the convexity of $\phi$ gives, relative to the arithmetic mean,
more weight to the high values in the sample. Those results are
reversed if $\phi$ is concave and increasing or convex and
decreasing. Jacks, Meissner and Novy's aggregation method of
the $\tau_{kj}$ is tantamount to using a $\phi$-mean with
$\phi$ the power functions $\phi: x \rightarrow
x^{\frac{1}{1-\sigma}}$, $\sigma>1$ where consistency with the
model imposes an arithmetic mean. Hence, since $\phi$ is
decreasing and convex, the substitution tends to underestimate trade costs.\\

This is a static result. We are interested in its consequences for the dynamic
behavior of trade flows. In terms of the
increase in trade flows, Jacks \textit{et al.}'s measure
overweights initially small trade partners. This can be seen by calculating the
elasticities to the importations from a country $k$ ($a_k$) of
the arithmetic mean ($m$) and of Jacks \textit{et al.}'s mean
($m_{\sigma}$) (keeping importations from other trade partners
$a_j$ constant). The elasticity of a generic mean $m_\sigma$ to
$a_k$ is given by
\mbox{$\varepsilon_{m_{\phi}}^k=\frac{1}{n}\frac{\phi'(a_k)}{\phi'(m_{\phi})}\frac{a_k}{m_{\phi}}$},
so that:

\begin{eqnarray*}
\varepsilon_m^k&=& \frac{1}{1+\sum\limits_{j \neq k} a_j/a_k}\\
\varepsilon_{m_{\sigma}}^k&=&\frac{1}{1+\sum\limits_{j\neq k} (a_j/a_k)^{\frac{1}{1-\sigma}}}
\end{eqnarray*}

The elasticity of the arithmetic mean $m$ to $a_k$ is
increasing in $a_k$ whereas the elasticity of $m_{\sigma}$ is
decreasing in $a_k$. A one-percent increase in the importations
from $k$ increases more the arithmetic mean of importations if
$k$ is initially an important importer, simply because $k$
represents a larger part of trade. But it is the opposite with
$m_{\sigma}$: the smaller the initial value of trade with $k$,
the bigger the impact of its growth on $m_{\sigma}$. Therefore,
Jacks \textit{et al.}'s measure is biased toward the growth rates of
costs incurred with initially small trade partners.

\bibliographystyle{aer}
\bibliography{bibliography}

\end{document}
